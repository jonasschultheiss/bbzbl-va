\chapter{Schlussbetrachtung}
\section{Von \bastian}
\authortoc{\bastian}{\sectionident}
Mir persönlich hat das Thema sehr gut gefallen. Es war auch mein Favorit unserer Vertiefungsarbeits-Ideen. Das Thema war sehr interessant und hatte einen riesigen inhaltlichen Umfang. Durch den Umfang waren uns keine Grenzen gesetzt. Ich kann mich auch persönlich mit dem Thema identifizieren, da ich Wert darauf lege, gesund zu leben. Ich denke, ich konnte auch die Ziele erfüllen, welche wir uns am Anfang gestellt haben. Mit dieser Vertiefungsarbeit habe ich einiges über die Gesundheit gelernt. Anhand des Projekts konnte ich gut sehen, wie es ist, wenn man auf verschiedene Faktoren achtet und rundum versucht, gesund zu leben. Die Erfahrung war sehr positiv mit der Projektumsetzung und ich will zukünftig fast alle Teile des Projekts beibehalten. Der Grund dafür ist, dass ich mich viel besser und wohler fühle, als ich es vor dem Projekt war. Was ich nicht weiterhin fortsetzen will, sind das Festhalten mithilfe des Fragebogens sowie dem Dokumentieren des Essens. 
\section{Von \dario}
\authortoc{\dario}{\sectionident}
DIESER ABSCHNITT IST NOCH NICHT VORHANDEN
\newline
\newline
\lipsum[4-8][32-64]
\section{Von \jonas}
\authortoc{\jonas}{\sectionident}
Ich finde, wir hatten ein gutes und interessantes Thema. Es hat mir Spass gemacht zu recherchieren und das Projekt durchzuziehen. Mit unseren circa 70 Seiten an geschriebenem haben wir die geforderten 30 deutlich überschritten. Ich denke, dies zeigt, wie umfänglich und interessant dieses Thema eigentlich ist.
\newline
\newline
Zudem denke ich, dass ich von dieser Arbeit sehr viele Erfahrungen in mein privates, berufliches und schulisches Leben übernehmen kann. Sich einmal in dieser Tiefe über dieses Thema informiert zu haben, kann einem für den Rest des Lebens eine bessere Perspektive bieten. Ausserdem wurde diese Vertiefungsarbeit mit LaTeX geschrieben. Durch diese “Übung“ konnte ich mein Wissen auch in diese Richtung vertiefen. Ich denke, dass mir dies für zukünftige Projekte, wie unter anderem der bevorstehenden individuelle praktische Abschlussprüfing, sehr helfen wird.
\newline
\newline
Ich möchte mich an dieser Stelle auch noch bei meinen beiden Teamkollegen für diese Erfahrung bedanken.