\chapter{Definition 'gesunder Lebensstil'}
\section{Ernährung}
\lipsum[4-8][32-64]
\section{Bewegung}
\lipsum[4-8][32-64]
\section{Schlaf}
\lipsum[4-8][32-64]
\section{Zeitmanagement}
Das Zeitmanagement wird von sehr vielen Menschen zu sehr unterschätzt. Viele Leben einfach von Tag zu Tag und lassen sich überraschen. Je mehr jedoch im Leben vorgeht, desto mehr geht vergessen, was vermeidbaren Stress entstehen lässt. 
\newline
Es empfiehlt sich deshalb sehr, zumindest einen Monats- und/oder Wochenplan zu erstellen, welchen man mit einer ToDo Liste verbinden kann. So kann schon viel Stress vermeidet werden, da man eine Übersicht über die kommende Woche und das Unerlidgte hat. 
\newline
Laut den Zahlen vom Bundesamt für Statistik von 2017 steigt die Anzahl der unter Stress leidenden immer mehr. 2012 gaben 18 Prozent der Befragten an, im Beruf aktiv unter Stress zu leiden. In der letzten Publikation von 2017 stieg dieser Wert auf 21 Prozent an. Ist eine Person über einen längeren Zeitraum Stress ausgesetzt, so kann sich das schlecht auf die mentale und physikale Gesundheit auswirken. Auf den Körper bezogen kann es zu Kopfschmerzen, Schlafproblemen, einem hohen Blutdruck und unter anderem zu einem schwächeren Immunsystem führen. Mental kommt es meist zu Ängsten, Wut, Gefühlen von überlastung, schwankende Laune, Konzentrationsproblemen, schlechtem Selbstbild und kann eine Depression verursachen. Auch das Verhalten einer Person kann sich anpassen. Symptome beim Verhalten sind über- oder unterernährung, Wutausbrüche, Probleme in Beziehungen und dem vermeiden von Personen. Zudem greifen Personen, welche einen längeren Zeitraum unter Stress leiden, ofters zu Suchtmitteln, wie Alkohol, Nikotin oder andere Drogen, als Personen welche verminderten oder gar keinen Stress haben.
\newline
https://www.bfs.admin.ch/bfs/de/home/aktuell/neue-veroeffentlichungen.assetdetail.9366230.html
\newline
https://www.healthdirect.gov.au/stress-symptoms
\section{Gewohnheiten}
\lipsum[4-8][32-64]
\section{Ergonomie am Arbeitsplatz}
\lipsum[4-8][32-64]