\chapter{Definition 'gesunder Lebensstil'}
Definition des WHOs: “Die Gesundheit ist ein Zustand des vollständigen körperlichen, geistigen und sozialen Wohlergehens und nicht nur das Fehlen von Krankheit oder Gebrechen.”
Einen gesunden Lebensstil beinhaltet nicht nur die Ernährung, sondern deckt auch einige andere Themenkreise ab, welche ansonsten nicht so oft thematisiert werden. Dazu haben wir einige Themen herausgesucht, welche unserer Meinung nach die Wichtigsten sind, um gesund zu leben. Diese Themen wären: Ernährung, Bewegung, Schlaf, Zeitmanagement und diverse Gewohnheiten. Auf diese werden wir nun im Detail eingehen.
\section{Ernährung}
Die Ernährung ist die Aufnahme von Nahrungsmittel, welche unseren Körper die notwendige Energie zur verfügung stellt. Somit spielt die Ernährung eine wichtige Rolle bei einem gesunden Lebensstil. Wenn man nicht auf die Ernährung achtet, zum Beispiel zu viel Fette oder zu wenig Vitamine, kann dies schlimme folgen haben. Dies kann zu Herz- Kreislaufstörungen, über- /untergewicht, Diabetes Typ 2 und vieles mehr führen.
\newline
Für die Definition, was eine gesunde Ernährung bedeutet beziehen wir uns auf die Schweizer ernährungspyramide.
\begin{figure}[!ht]
  \centering
  \includegraphics[width=0.5\linewidth]{./images/lebensmittelpyramide.png}
  \caption{Die Schweizer Lebensmittelpyramide}
  \label{fig:pyramide}
  \captionsetup{font={footnotesize}}
  \caption*{\url{https://www.sge-ssn.ch/ich-und-du/essen-und-trinken/ausgewogen/schweizer-lebensmittelpyramide/}}
\end{figure}
\begin{enumerate}
  \item \textbf{Getränke} - Es wird empfohlen 1-2 Liter Flüssigkeit pro Tag zu sich nehmen. Diese Getränke sollten am Besten ungesüsst sein. Mögliche Getränke wären zum Beispiel Wasser, Tee oder Kaffe. \cite{stufe_getraenke}
  \item \textbf{Gemüse und Früchte} - Es wird empfohlen 1-2 Liter Flüssigkeit pro Tag zu sich nehmen. Diese Getränke sollten am Besten ungesüsst sein. Mögliche Getränke wären zum Beispiel Wasser, Tee oder Kaffe. \cite{stufe_gemuese_fruechte}
  \item \textbf{Getreideprodukte, Kartoffeln und Hülsenfrüchte} - Es wird empfohlen täglich 3 Portionen aus dieser Stufe zu konsumieren. Die Grösse der Portionen ist Nahrungsmittel abhängig. Zum Beispiel 180g - 300g Kartoffeln oder 75g- 125g Brot.
  \newline
  Mit dieser Stufe nimmt man hauptsächlich Kalorien auf, was ein wichtiger Energielieferant für Muskeln, Gehirn und vieles mehr ist. \cite{stufe_getreideprodukte_kartoffeln_huelsenfruechte}
  \item \textbf{Milchprodukte, Fleisch, Fisch, Eier und Tofu} - Diese Stufe wurde in zwei Gruppen unterteilt: 
    \begin{enumerate}
    \item \textbf{Milchprodukte} - Es wird empfohlen täglich 3 Portionen Milch und Milchprodukte zu konsumieren. Die Grösse der Portionen ist auch wieder Nahrungsmittel abhängig. Zum Beispiel 2dl Milch oder 150g - 200g Joghurt/Quark usw.
    \newline
    Mit diesen Nahrungsmitteln nehmen wir Proteine und Calcium auf. Diese Stoffe sind wichtig für unsere Muskeln, Immunsystem, Knochen und vieles mehr. \cite{stufe_milch_milchprodukte}
    \item \textbf{Fleisch, Fisch Eier und Tofu} - Es wird empfohlen täglich 1 Portion Fleisch, Fisch, Eier oder Tofu zu konsumieren. Die Grösse der Portion ist auch wieder Nahrungsmittel abhängig. Zum Beispiel 2 - 3 Eier oder 100g - 120g Fleisch.
    \newline
    Mit den 3 Portionen Milchprodukten ist der tägliche Proteinbedarf noch nicht gedeckt. 
    \newline
    Daher wird zusätzlich eine Portion proteinreiches Lebensmittel empfohlen. Diese kann jedoch auch mit einer vierten Portion Milchprodukte umgangen werden. \cite{stufe_fleisch_fisch_eier_tofu_2}
  \end{enumerate}
  \item \textbf{Öle, Fette und Nüsse} - Es wird empfohlen täglich 20g - 30g Pflanzenöl zu konsumieren. Davon sollte die hälfte Rapsöl sein. Pro Tag sollte zusätzlich 20g - 30g ungesalzene Nüsse, Samen oder Kerne konsumiert werden.
  \newline
  Butter, Margarine, Rahm etc sollte nur sparsam verwendet werden.
  \newline
  Diese Stufe liefert viele Kalorien, lebensnotwendige Fettsäuren und fettlösliche Vitamine. \cite{stufe_le_fette_nuesse}
  \item \textbf{Süsses, Salziges und Alkoholisches} - Aus Ernährungssicht ist diese Stufe nicht notwendig. Jedoch für einen gesunden Lebensstil durchaus berechtigt.
  \newline
  Täglich sollte nur eine kleine Portion Lebensmittel aus dieser Stufe konsumiert werden.
  \newline
  Auch light- und zero-Softgetränke zählen zu dieser Stufe. Sie beinhalten zwar wenig Kalorien, jedoch gewöhnt man sich dadurch an den süssen Geschmack und die Säuren machen die Zähne kaputt. \cite{stufe_suesses_salziges_alkoholisches}
\end{enumerate}
Die Übertragung der Ernährungspyramide auf eine Mahlzeit kann man mit der nächsten Grafik einfach veranschaulichen.
\begin{figure}[!ht]
  \centering
  \includegraphics[width=0.5\linewidth]{./images/der_perfekte_teller.png}
  \caption{Optimaler Teller}
  \label{fig:teller}
  \captionsetup{font={footnotesize}}
  \caption*{\url{https://www.sge-ssn.ch/media/Merkblatt_der_optimale_Teller.pdf}}
\end{figure}
Dabei würde der Gemüse und Früchte Teil die Hälfte des Tellers einnehmen. Dafür nimmt man weniger Getreideprodukte, Kartoffeln oder Hülsenfrüchte zu sich.
\section{Bewegung}
Das BAG macht darauf aufmerksam, dass jede körperliche Aktivität sehr gut für die Leistungsfähigkeit und die Gesundheit ist. Bewegung in regelmässigen Abständen ist sehr wichtig um das Risiko für weit verbreitete Beschwerden und Krankheiten wie z.B. Übergewicht, Bluthochdruck, Herz-Kreislauf-Erkrankungen und Diabetes 2 zu reduzieren.
\newline
Wer sich viel bewegt lebt länger. Die Definition von Bewegung ist laut BAG “jede Form der Bewegung, die eine Anspannung der Muskeln erfordert und den Energieverbrauch im Vergleich zum Ruhezustand erhöht.”\cite{gesundheit_definition}
\newline
Ebenfalls verbessert die Bewegung die Lebensqualität, indem sie sich positiv auf das psychische Wohlbefinden auswirkt. Dies hängt mit Hormonen zusammen. Laut CSS lässt jede Art von Bewegung im Gehirn gewisse Hormone ausschütten, welche alle eine andere Auswirkung auf die Menschen haben.\cite{hormone-bei-bewegung} Es gibt zum einen die Ausschüttung von Stresshormonen, zum anderen schüttet der Körper auch Glückshormone aus. Bei den Stresshormonen spielen Cortisol sowie Adrenalin und Noradrenalin eine Rolle, damit der Körper genug Energie zugeführt bekommt. Bei den Glückshormonen geht man davon aus, dass Dopamin sowie Serotonin einen Glückszustand auslöst und dabei das Schmerzempfinden herabstuft.
\begin{figure}[!ht]
  \centering
  \includegraphics[width=0.38\linewidth]{./images/bewegungsempfehlungen-ew-dt.png}
  \caption{Grafik von BAG zur Bewegungsempfehlung von Erwachsenen Menschen.}
  \label{fig:bewegungsempfehlungen}
  \captionsetup{font={footnotesize}}
  \caption*{\url{https://www.bag.admin.ch/bag/de/home/gesund-leben/gesundheitsfoerderung-und-praevention/bewegungsfoerderung/bewegungsempfehlungen.html}}
\end{figure}
\newline
Das BAG empfiehlt für jeden Erwachsenen idealerweise mindestens 2h und 30min Bewegung pro Woche in Form von Alltagsaktivitäten oder Sport mit mittlerer Intensität.\cite{bewegungsfoerderung} Dazu gehören zum Beispiel das Velofahren, das Gehen oder Gartenarbeit. Diese Empfehlung kann auch durch Sport mit hoher Intensität ersetzt werden. Dabei wird mindestens 1h und 15min Bewegung pro Woche als Basiswert genommen. Sportarten mit hoher Intensität sind zum Beispiel Rennen, Krafttraining, Skifahren und noch diverse Andere.
\section{Schlaf}
“Der Schlaf ist eine grundlegende biologische Funktion und für das Wohlbefinden eines Menschen notwendig” \cite{bundesamtfrstatistik_2015_schlafstoerungen}. So lautet die Definition vom Bundesamt für Statistik.
\newline
Der Schlaf sorgt für eine gute Lebensqualität indem er sämtliche Kräfte wiederherstellt. Laut Gesundheit ist Schlaf auch bei weitem keine passive Tätigkeit, Der Körper reagiert schwächer, jedoch finden wichtige Auf- und Abbauprozesse statt. Für einen gesunden Lebensstil sind zwei Phasen des Schlafens sehr wichtig. Dazu gehören die Tiefschlafphase, sowie die Traumphase. In der Tiefschlafphase erholt sich der Körper und “repariert” unsere Organe. Die Traumphase ist wichtig für die psychische Erholung. Es gibt einige Menschen in der Schweiz, welche an Schlafstörungen leiden. Laut dem BFS leidet ein Viertel der Bevölkerung an Schlafstörungen. Diese haben zur Folge, dass man physische und psychische Gesundheitsprobleme erlangen kann oder das Diabetes, sowie Adipositas Risiko steigt. Einen Mangel an Schlaf stört ebenfalls das Gedächtnis \cite{schlaf-grundbeduerfnis-und-lebenselixier}. Während des Schlafens wird gelerntes und Erfahrungen verarbeitet und zugeordnet. Bei ständigem Mangel ist die Funktion des Gehirns eingeschränkt, sodass es nicht alles richtig verarbeiten kann. Der Schlaf hat weitere starke Einflüsse. Zum einen stärkt er das Immunsystem und zum anderen reguliert er den Hunger. Wenn man zu wenig schläft ist man somit angreifbarer für diverse Infektionen. Ebenfalls wird das Gleichgewicht zwischen den Hormonen für Hunger stark gestört, was zu einem grösseren Appetit führt.
\section{Zeitmanagement}
Das Zeitmanagement wird von sehr vielen Menschen zu sehr unterschätzt. Viele Leben einfach von Tag zu Tag und lassen sich überraschen. Je mehr jedoch im Leben vorgeht, desto mehr geht vergessen, was vermeidbaren Stress entstehen lässt. 
\newline
Es empfiehlt sich deshalb sehr, zumindest einen Monats- und/oder Wochenplan zu erstellen, welchen man mit einer ToDo Liste verbinden kann. So kann schon viel Stress vermeidet werden, da man eine Übersicht über die kommende Woche und das Unerlidgte hat. 
\newline
Laut den Zahlen vom Bundesamt für Statistik von 2017 steigt die Anzahl der unter Stress leidenden immer mehr. 2012 gaben 18 Prozent der Befragten an, im Beruf aktiv unter Stress zu leiden. In der letzten Publikation von 2017 stieg dieser Wert auf 21 Prozent an. Ist eine Person über einen längeren Zeitraum Stress ausgesetzt, so kann sich das schlecht auf die mentale und physikale Gesundheit auswirken. Auf den Körper bezogen kann es zu Kopfschmerzen, Schlafproblemen, einem hohen Blutdruck und unter anderem zu einem schwächeren Immunsystem führen. Mental kommt es meist zu Ängsten, Wut, Gefühlen von überlastung, schwankende Laune, Konzentrationsproblemen, schlechtem Selbstbild und kann eine Depression verursachen. Auch das Verhalten einer Person kann sich anpassen. Symptome beim Verhalten sind über- oder unterernährung, Wutausbrüche, Probleme in Beziehungen und dem vermeiden von Personen. Zudem greifen Personen, welche einen längeren Zeitraum unter Stress leiden, ofters zu Suchtmitteln, wie Alkohol, Nikotin oder andere Drogen, als Personen welche verminderten oder gar keinen Stress haben.
\newline
https://www.bfs.admin.ch/bfs/de/home/aktuell/neue-veroeffentlichungen.assetdetail.9366230.html
\newline
https://www.healthdirect.gov.au/stress-symptoms
\section{Gewohnheiten}
\lipsum[4-8][32-64]
\section{Ergonomie am Arbeitsplatz}
Schon als Kind wird uns in der Schule beigebracht, dass wir richtig am Tisch sitzen sollen. Bei uns kam dies ab und zu nicht ganz an, denn bequem war es nicht unbedingt.
\newline
Doch der Raht der Lehrer ist berechtigt, auch wenn uns dies damals nicht ganz klar war. Wenn man es gewohnt ist über einen längeren Zeitraum fallsch zu sitzen, riskiert die Gesundheit seines Rückens. So kann es sein, dass man Rückenprobleme bekommnt, die einen bis ans Lebensende heimsuchen.
\newline
\newline
Es können allerdings auch andere Probleme auftreten:
\begin{itemize}
  \item Gewebe-Irritationen
  \item Atembeschwerden
  \item Abklemmen der Nerven und Adern
  \item Verdauungsstörungen
  \item Probleme mit dem Aufstehen
  \item Muskelschmerzen Gelenk-Probleme
\end{itemize}
(https://www.fitform-sessel.de/sitzen/die-folgen-von-falschem-sitzen/weitere-folgen-von-falschem-sitzen)
\newline
Deshalb ist es wichtig, schon früh aus dem richtigen Sitzen eine Gewohnheit zu machen. Rückenschmerzen sind sehr unangenehm und können vorallem im erhöhten Alter zu grossen Problemen führen. Ist der Rücken einmal beschädigt, kann man nicht wirklich mehr zum alten Stand zurück. Man kann zwar die Symptome wie zum Beispiel schmerzen behandeln, doch gegen die Ursache selbst ist kann nur in wenigen Fällen etwas unternommen werden.
\newline
Das richtige Sitzen ist vorallem bei Berufen wichtig, wo die Angestellten die meiste Arbeitszeit sitzend verbringen. 
\newline
Doch wie sitz man richtig?
\subsection{Reflexionen und Blendungen}
Sehr oft wird korrigieren wir unsere Sitzposition, wenn die Sonne auf unseren Bildschirm scheint. Dies ist auf lange Sicht nicht sinnvoll und sollte angepasst werden. Um Reflexionen zu vermeiden, kann man den Bildschirm so positionieren, dass das Sonnenlicht nicht direkt auf den Bildschirm scheint. Dafür kann man den Bildschirm 90 Grad zum Fenster stellen. Dies sollte die Reflexion vermeiden, beziehunngweise einschränken.
\newline
In manchen Fällen reicht dies nicht. In solchen Fällen werden oft Storen genutzt. Dies ist nicht optimal, da Storen immernoch Lichtstreifen durchlassen und/oder den Raum zu sehr abdunkeln. Was stattdessen empfohlen wird sind Folienrollos oder Lamellenvorhänge mit vertikalen Streifen. Diese vermeiden Reflexionen besser, dunkeln aber den Raum nicht so sehr ab wie heruntergelassene Storen.
https://www.suva.ch/de-CH/material/Factsheets/arbeitsplatz-einrichten
\section{Definition für verschiedene Menschen}