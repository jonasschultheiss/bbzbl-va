\chapter{Projekt}

\section{Einführung}
Neben der fachlichen Auseinandersetzung mit dem Thema “gesunder Lebensstil”, haben wir ein Projekt durchgeführt.
\newline
In diesem Projekt wollen wir selbst erfahren, was wir für Veränderung, sowie Unterschiede bemerken, wenn wir gesund leben.
\newline
Die \textit{“Einschränkungen”} für ein gesunden Lebensstil, haben wir anhand, der vorherigen Kapiteln bestimmt. Zusätzlich haben wir versucht, diese möglichst alltagstauglich zu definieren. Auf die \textit{“Einschränkungen”} gehen wir später noch genauer ein.
\newline
Dieses Projekt haben wir insgesamt vier Wochen im Zeitraum vom \textit{21.10.2020} bis \textit{17.11.2020} durchgeführt.
\subsection{Vorheriger Lebensstil}
\subsubsection{Von Dario Grob}
Wie schon im Kapitel \ref{bezug:dario} angesprochen, habe ich über eine länger Zeit nicht auf einen gesunden Lebensstil geachtet. In den letzten Sommerferien haben ich beschlossen, mich regelmässiger sportlich zu betätigen und einen Schlafrhythmus zu etablieren.
\newline
In den Herbstferien bin ich jedoch in alte Muster zurückgefallen. Ein normaler Arbeitstag hat folgendermassen ausgesehen.
\newline
Ich gehe ca. um 02:45 Uhr ins Bett und muss um 08:15 Uhr aufstehen. Somit schlafe ich durschnittlich 5h 30min pro Nacht. Dadurch bin den ganzen Tag sehr müde und trinke viel Energydrinks und Kaffee um mich wach zu halten. 
\newline
Über den Mittag esse ich auswärts. Dies ist oft Fast Food und eher ungesund.
\newline
Bis um 18:30 Uhr arbeite ich und gehe daraufhin nach hause. Das Abendessen Zuhause ist oft gesund, da meine Mutter kocht und auf eine gesunde Ernährung achtet. 
\newline
In meiner Freizeit game ich sehr viel oder schaue Serien und bewege mich somit nicht viel. 
\newline
Die Hausaufgaben, das Lernen auf Tests, sowie das Arbeiten an Projekten schiebe ich solang es geht vor mir her. Dadurch bin ich vor den Schultagen sehr in stress.
\newline
Den ganzen Tag durch mache ich regelmässig Pausen und rauche dazu eine Zigarette oder nehme ein Snus. Durschnittlich rauche ich am Tag 10 Zigaretten und nehme 3 - 4 Snus.
\newline
Durch meinen Lebensstil habe ich negative folgen. Ich habe oft starke Rückenschmerzen durch das viele Sitzen und der fehlenden Bewegung, bin oft sehr müde dadurch, dass ich sehr wenig schlafe und nehme in letzter Zeit an Gewicht zu.
\subsubsection{Von Bastian Büeler}
Früher habe ich immer wenig geschlafen und bin ca. zwischen 24:00-02:00 Uhr schlafen gegangen. Weil ich zu wenig Nachtruhe hatte und mein Körper viel mehr Schlaf gebraucht hätte, war ich am Folgetag oftmals sehr müde. Ebenfalls habe ich immer wieder ungesundes Essen zu mir genommen, wie zum Beispiel Pizza. Zudem habe ich nicht auf die Ernährungspyramide geachtet und einfach irgendwie gegessen. Dazu kam noch die Müdigkeit, welche zusätzlich hungrig machte. Die Aufgaben habe ich mir im Kopf gemerkt und keine ToDo Liste oder sonstiges erstellt und die Bewegung hat seit der Corona Krise ebenfalls stark abgenommen. Alles in allem würde ich sagen, dass mein vorheriger Lebensstil nicht wirklich gesund war.
\subsubsection{Von Jonas Schultheiss}
Ich habe dieses Thema schon in Kapitel \ref{bezug:jonas} angesprochen.
\newline
Mitte bis Ende der Sekundarschule habe ich mich etwas gehen lassen. Dies betrifft das ganze Spektrum von Aspekten, in welchen es in unserer Vertiefungsarbeit geht. Sport, Schlaf und so weiter. Nachdem ich es im Unihockeyverein nicht in die U18A geschaft habe, hatte ich keine Lust mehr auf Unihockey. Denn mir blieb nur die Wahl in einen deutlich schlechteren Verein zu wechseln, wo ich niemanden kannte und einen längeren Weg hätte, oder aufzuhören. Dannach machte ich eine Pause vom Sport, bis ich ein Jahr später mit Volleyball in Therwil begann. Dies zog ich zwei Jahre durch, bis ich schlussendlich aufhörte, wegen meiner Körpergrösse (173cm) und weil ich mich meiner Meinung nach zu langsam verbesserte. Ein halbes Jahr später gab ich Unihockey nochmals eine Chance. Ein paar meiner Kollegen sind in einem Verein in Flüh und nahmen mich mit. Ich spielte für ein Semester, bis aufhörte. Ich hatte sehr Probleme mitzuhalten, da ich mit 16 die grandiose Idee hatte, mit dem rauchen anzufangen. In meinem alten Verein war ich noch einer der Besten, wenn es um die Ausdauer ging. Meine Position auf dem Feld war sogar die, die am meisten Aussdauer benötigte. Nun, 3 Jahre später konnte ich nun nur noch zehn Minuten voll mitspielen, bevor meine Lunge zu schmerzen began und ich mich am liebsten übergäben hätte.
\newline
Wie schon angesprochen habe ich im Januar 2017 angefangen zu rauchen. Vor dem Begin der Lehre habe ich zwei Monate lang aufgehört und dachte mir, ich würde nie wieder rauchen. Anfangs der Sommerferien hatte ich allerdings einen \textit{sehr} schlechten Tag, worrauf hin ich wieder began. Seither habe ich unzählige Male probiert aufzuhören. Doch habe es bisher nie komplett geschaft. Nach zwei Tagen habe ich extreme Konzentrationsprobleme und werde durch Kleinigkeiten sehr aggressiv. Da diese zwei Aspekte in der Berufsschule oder im Betrieb eher störend sind, habe ich es immer auf die Ferien gelegt. Die Situation mit der Lunge in verbindung mit Sport brachte mich zum Umstieg von Zigarette zu Snus. Ich denke nicht, dass Snus im Allgemeinen weniger schädlich ist, doch ich kann nun aus Erfahrung sagen, dass sich meine Lunge regeneriert hat. Ich kann nun wieder relativ Problemlos joggen gehen.
\newline
Nun würde ich gerne zu meinem Konsum von koffeinhaltigen Getränken kommen. Während beziehungsweise nach meinen extremen Schlafproblemen ist dieser Konsum geradezu explodiert. Zuvor habe ich an manchen Tagen 250ml zu mir genommen. Doch während und nachdem ist es meist an jedem Tag die gleiche bis die dreifache Menge. Ich war am morgen komplett unbrauchbar ohne das Koffein, weil ich während den Problemen oft nur drei bis sechs Stunden schlief. Dann konnte ich das Problem mit Schlaftabletten in den Griff bekommen. Durch diese konnte ich zwar einschlafen, doch war am Morgen genau so müde wie zuvor. Mittlerweile nehme ich die Tabletten nicht mehr. Doch aus dieser Phase ist nun eine Gewohnheit geworden. Trinke ich mal nicht eines, fühle ich mich unkonzentriert und müde, obwohl ich genug geschlafen habe. Ich fühle mich nicht so müde wie während dem Problem, sondern ich spüre einfach, dass mir etwas fehlt.
\subsection{Was machen wir?}
\subsubsection{Ernährung}
In unserem Projekt achten wir auf unser Ernährung. Dabei richten wir uns nach der Definition aus dem Kapitel “Ernährung”.
\newline
Unser Ernährung dokumentieren wir in der App “Lifesum”. In dieser App sind “alle” Lebensmittel und Nahrungsmittel hinterlegt. Dazu liefert die App eine kurze Zusammenfassung, ob ein Nahrungsmittel gesund ist und was für Inhaltsstoffe dies hat. Neben dem Dokumentieren unserer Ernährung, konnten wir uns an Rezeptvorschläge der App orientieren, falls wir etwas kochen mussten.
\subsubsection{Bewegung}
\subsubsection{Schlaf}
\subsubsection{Zeitmanagement}
\subsubsection{Gewohnheiten}
\subsubsection{Bullet Journaling}
\section{Auswertung}
\section{Unsere Erfahrungen im Projekt}