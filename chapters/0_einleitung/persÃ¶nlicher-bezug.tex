\chapter{Persönlicher Bezug zum Thema}
\section{Zum Oberthema}
\authortoc{\dario}{\sectionident}
Momentan achtet keiner aus unserer Gruppe auf einen umfassenden gesunden Lebensstil.
Um einen gesunden Lebensstil zu führen muss man auf diverse Sachen achten und einige “Einschränkungen” vornehmen. Trotzdem achten viele Menschen auf einen gesunden Lebensstil und sind von den Auswirkungen überzeugt. 
Somit glauben wir, dass ein gesunder Lebensstil sich stark von unserem Lebensstil unterscheidet. Deswegen würden wir unser Thema “gesunder Lebensstil” dem Oberthema “Leben mit ….” unterordnen.
\section{Zu unserem Thema}
\subsection{Bezug von \bastian}
\authortoc{\bastian}{\subsectionident}
Ich befasse mich seit einiger Zeit mit gesunder Ernährung und Sport. Dieses Thema interessiert mich sehr, da ich mich im gesamten mit einem gesunden Lebensstil befassen möchte. Dies ist bis anhin nicht in allen Bereichen der Fall gewesen. Ein Beispiel dafür wäre der Schlaf gewesen. Diesen habe ich immer vernachlässigt und ich bin erst sehr spät schlafen gegangen. Das hat dazu geführt, dass ich am nächsten Tag sehr müde gewesen bin. Deshalb möchte ich es nun richtig umsetzen, sodass ich von einem vollumfänglichen gesunden Lebensstil sprechen kann. Ich bin gespannt, wie gut ich das schlussendlich umsetzen kann.
\subsection{Bezug von \dario}
\authortoc{\dario}{\subsectionident}
\label{bezug:dario}
Vor der Lehre habe ich relativ gesund gelebt. Mit dem Beginn der Lehre, habe ich jedoch angefangen immer weniger auf ein gesunden Lebensstil zu achten.
\newline
Ich habe immer weniger Sport gemacht, habe öfters auswärts ungesund gegessen, habe angefangen zu rauchen und vieles mehr. 
\newline
Die ersten zwei Jahre hatte ich eigentlich keine Auswirkungen von meinem ungesunden Lebensstil. 
\newline
Das letzte Jahr über, habe ich jedoch langsam die ersten Auswirkungen bemerkt. Ich hatte oft Rückenschmerzen, war durch den Tag immer müde und unkonzentriert und vieles mehr.
\newline
In den letzten Sommerferien habe ich dann beschlossen, dass ich etwas ändern muss. Ich habe beschlossen, mehr Sport zu treiben und auf meinen Schlafrhythmus zu achten. 
\newline
Dies hat zu beginn sehr gut funktioniert, jedoch habe ich über die Zeit immer mehr die Motivation verloren.
\newline
Ich hoffe nun, dass ich mit dieser Arbeit wieder mehr Motivation finde und meinen Lebensstil für eine längere Zeit anpassen kann.
\subsection{Bezug von \jonas}
\authortoc{\jonas}{\subsectionident}
\label{bezug:jonas}
Ich bin ein eher negatives Beispiel, wenn es um einem gesunden Lebensstil geht. Ich war zu meiner Sekundarschulzeit jedoch noch recht sportlich. Ich hatte mehrmals pro Woche Training und ging auch sonst gerne in meiner Freizeit mal joggen. Nachdem ich es nicht in die U18 Mannschaft des UBR's geschafft habe, habe ich dies allerdings etwas schleifen lassen.
\newline
Dies blieb allerdings konstant bis zum Herbst 2019. Seit vergangenen Frühling habe ich mich schon mit diesem Thema aktiv auseinandergesetzt. Ich konnte durch den Frühling und den Sommer zum Beispiel meinen Schlafrhythmus einpendeln, etwas, was mir seit Mitte der Sekundarschule Probleme gemacht hat. Ausserdem fing ich an mit 'Bullet Journaling', womit ich auch einiges an Stress aus meinem Leben genommen habe. Einzig mit der Nikotinsucht und dem Konsum von Zucker und Koffein habe ich meine Probleme. Klar, ich könnte auch siebenmal die Woche Sport machen, allerdings ist es mir momentan wichtiger, den Konsum in den Griff zu bekommen.
\newline
\newline
Als Bastian dieses Thema vorgeschlagen hat, war ich sehr interessiert. Ich stellte mir nur noch die Frage, ob ich die Aspekte mit der Ernährung und dem Sport auf längere Zeit durchziehen könne.