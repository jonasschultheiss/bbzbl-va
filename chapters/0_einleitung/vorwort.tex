\chapter{Vorwort}
\authortoc{\dario}{\chapterident}
Wir befassen uns in dieser Vertiefungsarbeit mit dem Thema “gesunder Lebensstil”. Ein gesunder Lebensstil kann das Leben massgeblich verändern. Er kann Krankheiten vorbeugen und das Körpergefühl oder die Psyche beeinflussen.
\newline
Immer mehr Menschen achten darauf, dass sie ein “gesundes” Leben führen. Dies merken wir vor allem in unserem sozialen Umfeld oder auf Social Media. Dadurch ist über die Zeit ein grosser Markt mit viel Umsatz entstanden. Jedoch gibt es keine klaren Antworten auf die Frage “Was ist gesund?”. Für jede Person kann dies anders aussehen. Dies hat dazu geführt, dass es viele Falschinformationen und Scharlatane im Internet gibt. Es gibt aber auch viele wissenschaftliche Arbeiten, welche sich mit dem Thema tiefgründig befassen.
\newline
Wir möchten darauf hinweisen, dass wir keine Experten in diesem Gebiet sind und somit hier keine wissenschaftliche Arbeit schreiben. Wir haben weder das medizinische Fachwissen noch andere tiefgründige Vorkenntnisse zu diesem Thema. Trotzdem werden wir uns über das Thema “gesunder Lebensstil” informieren und auf möglichst vertrauenswürdige Quellen zurückgreifen.
\newline
Gesund Leben kann man grundsätzlich in zwei Gebiete aufteilen. Einerseits die psychische Gesundheit, andererseits die körperliche Gesundheit. Da dies zwei sehr grosse Themengebiete sind, haben wir diese in fünf Themen unterteilt: Ernährung, Bewegung, Zeitmanagement, Schlaf und Gewohnheiten. Wieso wir diese gewählt haben, erklären wir später beim Mindmap.