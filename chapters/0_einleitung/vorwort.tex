\chapter{Vorwort}
Wir befassen uns in dieser Vertiefungsarbeit mit dem Thema “Gesunder Lebensstil”.
\newline
Ein gesunder Lebensstil kann das Leben massgeblich verändern. Es kann Krankheiten vorbeugen und das Körpergefühl oder die Psyche beeinflussen.
\newline
Immer mehr Menschen achten darauf, dass sie ein “gesundes” Leben führen. Dies merken wir vor allem in unserem Sozialen Umfeld oder auf Social Media.
\newline
Dadurch ist über die Zeit ein grosser Markt mit viel Umsatz entstanden.
\newline
Jedoch gibt es keine klare Antworten auf die Frage “Was ist Gesund?”. Für jede Person kann dies anders aussehen. 
\newline
Dies hat dazu geführt, dass es viele Falschinformationen und Scharlatane im Internet gibt.
\newline
Es gibt aber auch viele Wissenschaftliche Arbeiten, welche sich mit dem Thema tiefgründig befassen.
\newline
Wir möchten darauf hinweisen, dass wir keine Experten in diesem Gebiet sind und somit hier keine Wissenschaftliche Arbeite schreiben. Wir haben weder das medizinische Fachwissen als auch keine anderen tiefgründigen Vorkenntnisse zu diesem Thema.
\newline
Trotzdem werden wir uns über das Thema “Gesunder Lebensstil” informieren und auf möglichst vertrauenswürdige Quellen zurückgreifen.
\newline
Gesund Leben kann man grundsätzlich in zwei gebiete aufteilen. Einerseits die psychische Gesundheit andererseits die körperliche Gesundheit. Da dies zwei sehr grosse Themengebiete sind, haben wir dies in fünf Unterthemen unterteilt.
\newline
(Ernährung, Bewegung, Zeitmanagement, Schlaf und Gewohnheiten). Wieso wir diese gewählt haben, erklären wir später beim Mindmap.
