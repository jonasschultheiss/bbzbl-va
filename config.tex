%%%%%%%%%%%%%%%%%%%%%%%%%%%%%%%%%%%%%%%%%
% This is the LaTeX template of Jonas Schultheiss. It'll be used for the IPA and VA.
% 
% This template can be downloaded here:
% https://github.com/
% 
% This template is heavily inspired by Marco Roth's template:
% https://github.com/marcoroth
%
% Template license:
% MIT
% https://github.com/jonasschultheiss/LaTeX-Template/blob/master/LICENSE
% ./LICENSE
%%%%%%%%%%%%%%%%%%%%%%%%%%%%%%%%%%%%%%%%%

%%%%%%%%%%%%%%%%%%%%%%
% Document & Variables
%%%%%%%%%%%%%%%%%%%%%%

\documentclass[10pt, oneside]{scrbook}

\newcommand{\version}{0.1}
\newcommand{\docdate}{\today}
\newcommand{\dario}{Dario Grob}
\newcommand{\bastian}{Bastian Büeler}
\newcommand{\jonas}{Jonas Schultheiss}
\newcommand{\alle}{\bastian, \dario, \jonas}
\newcommand{\docname}{Alltagstauglicher gesunder Lebensstil}
\newcommand{\compiledfilename}{Vertiefungsarbeit-DG-BB-JS.pdf}
\newcommand{\docauthor}{\alle}
\newcommand{\task}{Vertiefungsarbeit 2020} % e.g. Modul 152/IPA/...
\newcommand{\clientlogopath}{./images/BBZBL_logo.png}




%%%%%%%%%%
% Packages
%%%%%%%%%%

\usepackage[
  a4paper,
  left=25mm,
  right=25mm,
  top=35mm,
  bottom=35mm,
  headheight=35mm
]{geometry}
\usepackage[utf8]{inputenc}
\usepackage[babel,german=swiss]{csquotes}
\usepackage[ngerman]{babel}
\usepackage[printonlyused]{acronym}
\usepackage[section]{placeins}
\usepackage[Conny]{fncychap}
\usepackage[T1]{fontenc}
\usepackage[T1]{fontenc}
\usepackage[table]{xcolor}
\usepackage{caption}
\usepackage{color}
\usepackage{fancyhdr}
\usepackage{graphics}
\usepackage{graphicx}
\usepackage{helvet}
\usepackage{lastpage}
\usepackage{lipsum}
\usepackage{listings}
\usepackage{multicol}
\usepackage{multirow}
\usepackage{parskip}
\usepackage{pdfpages}
\usepackage{pifont}
\usepackage{tabularx}
\usepackage{tocloft}
\usepackage{xurl}
\usepackage{wrapfig}
\usepackage{xcolor}
\usepackage{afterpage}
\usepackage[backend=biber, style=numeric]{biblatex}
\usepackage{suffix}
\usepackage{float}

%%%%%%%%
% Colors
%%%%%%%%

\definecolor{lightgray}{RGB}{224,224,224}
\definecolor{darkgray}{RGB}{192,192,192}

\renewcommand{\listfigurename}{List of figures}


%%%%%%%%%%%%%%%%%%%
% Headers & Footers
%%%%%%%%%%%%%%%%%%%

\pagestyle{fancy}
\renewcommand{\chapterpagestyle}{fancy}
\renewcommand{\partpagestyle}{fancy}
\fancyhead{}
\fancyhead[R]{\fontsize{10}{12} \selectfont \textbf{\task} \\ \fontsize{8}{10} \selectfont \docauthor \\ \docname}
\fancyhead[L]{\includegraphics[width=5cm]{\clientlogopath}}
\fancyfoot{}
\fancyfoot[L]{\fontsize{10}{11} \selectfont \compiledfilename}
\fancyfoot[C]{\fontsize{9}{11} \docdate}
\fancyfoot[R]{\fontsize{10}{11} \selectfont Seite \thepage\ von \pageref{LastPage}}

\renewcommand{\headrulewidth}{0pt}

%%%%%%
% Font
%%%%%%

\renewcommand{\familydefault}{\sfdefault}

%%%%%
% TOC
%%%%%

\renewcommand{\cftchapafterpnum}{\vspace{5pt}}

%%%%%%%%%%%%
% Empty Page
%%%%%%%%%%%%

\newcommand\blankpage{%
    \null
    \thispagestyle{empty}%
    \addtocounter{page}{-2}%
    \newpage}

%%%%%
% LOF
%%%%%

\captionsetup{justification=centering}

%%%%%%%%%%
% Graphics
%%%%%%%%%%

\setcounter{totalnumber}{5}
\ChNumVar{\Huge}
\ChTitleVar{\huge\sffamily}
\ChNameVar{\large\sffamily}
\ChRuleWidth{0.5pt}

%%%%%%%%%%%%%%
% Line spacing
%%%%%%%%%%%%%%

\renewcommand{\baselinestretch}{1,3}

%%%%%%%%%%%%%%%%
% PDF formatting
%%%%%%%%%%%%%%%%

\definecolor{codegray}{gray}{0.9}
\newcommand{\code}[1]{\colorbox{codegray}{\texttt{#1}}}
\usepackage[
  bookmarks=true,           % Lesezeichen erzeugen
  bookmarksopen=true,       % Lesezeichen ausgeklappt
  bookmarksnumbered=true,   % Anzeige der Kapitelzahlen
  breaklinks=true,          % Ermöglicht einen Umbruch von URLs
  colorlinks=true,          % Einfärbung von Links
  linkcolor=black,          % Linkfarbe: schwarz
  anchorcolor=black,        % Ankerfarbe: schwarz
  citecolor=black,          % Literaturlinks: schwarz
  filecolor=black,          % Links zu lokalen Dateien: schwarz
  menucolor=black,          % Acrobat Menü Einträge: schwarz
  urlcolor=black,           % URL-Farbe: schwarz
  pdftitle={Vertiefungsarbeit 2020},
  pdfauthor={Dario Grob, Bastian Büeler, Jonas Schultheiss},
  pdfsubject={Alltagstauglicher gesunder Lebensstil},
  pdfkeywords={VA, Vertiefungsarbeit, Gesunder, Lebensstil}
]{hyperref}

%%%%%%%%%%%%%%
% Reset indent
%%%%%%%%%%%%%%

\setlength{\parindent}{0pt}

%%%%%%%%%
% Columns
%%%%%%%%%

\setlength{\columnseprule}{1pt}
\def\columnseprulecolor{\color{black}}

%%%%%%%%%%%%%
% Titel fonts
%%%%%%%%%%%%%

\newcommand{\titlesize}{\fontsize{25pt}{20pt}\selectfont}
\newcommand{\subtitlesize}{\fontsize{15pt}{10pt}\selectfont}
\newcommand{\subsubtitlesize}{\fontsize{10pt}{5pt}\selectfont}

\newcommand\chapterauthor[2]{\authortoc{Von #1}{#2}\printchapterauthor{Von #1}}
\WithSuffix\newcommand\chapterauthor*[1]{\printchapterauthor{Von #1}}

\makeatletter\@openrightfalse
\newcommand{\printchapterauthor}[1]{%
  {\parindent0pt\vspace*{-60pt}%
  \linespread{1.1}\large\scshape#1%
  \par\nobreak\vspace*{35pt}}
  \@afterheading%
}
\newcommand{\authortoc}[2]{%
  \addtocontents{toc}{\vskip-10pt}%
  \addtocontents{toc}{%
    \protect\contentsline{chapter}%
    {\hskip#2\mdseries\scshape\protect\scriptsize Von #1}{}{}}
}
\makeatother

\newcommand{\chapterident}{1.55em}
\newcommand{\sectionident}{3.85em}
\newcommand{\subsectionident}{7em}
\newcommand{\subsubsectionident}{19em}

\renewcommand*{\partformat}{\partname~\thepart\autodot}